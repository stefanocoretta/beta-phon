% Options for packages loaded elsewhere
\PassOptionsToPackage{unicode}{hyperref}
\PassOptionsToPackage{hyphens}{url}
\PassOptionsToPackage{dvipsnames,svgnames,x11names}{xcolor}
%
\documentclass[
  authoryear,
  preprint,
  3p]{elsarticle}

\usepackage{amsmath,amssymb}
\usepackage{iftex}
\ifPDFTeX
  \usepackage[T1]{fontenc}
  \usepackage[utf8]{inputenc}
  \usepackage{textcomp} % provide euro and other symbols
\else % if luatex or xetex
  \usepackage{unicode-math}
  \defaultfontfeatures{Scale=MatchLowercase}
  \defaultfontfeatures[\rmfamily]{Ligatures=TeX,Scale=1}
\fi
\usepackage{lmodern}
\ifPDFTeX\else  
    % xetex/luatex font selection
\fi
% Use upquote if available, for straight quotes in verbatim environments
\IfFileExists{upquote.sty}{\usepackage{upquote}}{}
\IfFileExists{microtype.sty}{% use microtype if available
  \usepackage[]{microtype}
  \UseMicrotypeSet[protrusion]{basicmath} % disable protrusion for tt fonts
}{}
\makeatletter
\@ifundefined{KOMAClassName}{% if non-KOMA class
  \IfFileExists{parskip.sty}{%
    \usepackage{parskip}
  }{% else
    \setlength{\parindent}{0pt}
    \setlength{\parskip}{6pt plus 2pt minus 1pt}}
}{% if KOMA class
  \KOMAoptions{parskip=half}}
\makeatother
\usepackage{xcolor}
\setlength{\emergencystretch}{3em} % prevent overfull lines
\setcounter{secnumdepth}{5}
% Make \paragraph and \subparagraph free-standing
\makeatletter
\ifx\paragraph\undefined\else
  \let\oldparagraph\paragraph
  \renewcommand{\paragraph}{
    \@ifstar
      \xxxParagraphStar
      \xxxParagraphNoStar
  }
  \newcommand{\xxxParagraphStar}[1]{\oldparagraph*{#1}\mbox{}}
  \newcommand{\xxxParagraphNoStar}[1]{\oldparagraph{#1}\mbox{}}
\fi
\ifx\subparagraph\undefined\else
  \let\oldsubparagraph\subparagraph
  \renewcommand{\subparagraph}{
    \@ifstar
      \xxxSubParagraphStar
      \xxxSubParagraphNoStar
  }
  \newcommand{\xxxSubParagraphStar}[1]{\oldsubparagraph*{#1}\mbox{}}
  \newcommand{\xxxSubParagraphNoStar}[1]{\oldsubparagraph{#1}\mbox{}}
\fi
\makeatother

\usepackage{color}
\usepackage{fancyvrb}
\newcommand{\VerbBar}{|}
\newcommand{\VERB}{\Verb[commandchars=\\\{\}]}
\DefineVerbatimEnvironment{Highlighting}{Verbatim}{commandchars=\\\{\}}
% Add ',fontsize=\small' for more characters per line
\usepackage{framed}
\definecolor{shadecolor}{RGB}{241,243,245}
\newenvironment{Shaded}{\begin{snugshade}}{\end{snugshade}}
\newcommand{\AlertTok}[1]{\textcolor[rgb]{0.68,0.00,0.00}{#1}}
\newcommand{\AnnotationTok}[1]{\textcolor[rgb]{0.37,0.37,0.37}{#1}}
\newcommand{\AttributeTok}[1]{\textcolor[rgb]{0.40,0.45,0.13}{#1}}
\newcommand{\BaseNTok}[1]{\textcolor[rgb]{0.68,0.00,0.00}{#1}}
\newcommand{\BuiltInTok}[1]{\textcolor[rgb]{0.00,0.23,0.31}{#1}}
\newcommand{\CharTok}[1]{\textcolor[rgb]{0.13,0.47,0.30}{#1}}
\newcommand{\CommentTok}[1]{\textcolor[rgb]{0.37,0.37,0.37}{#1}}
\newcommand{\CommentVarTok}[1]{\textcolor[rgb]{0.37,0.37,0.37}{\textit{#1}}}
\newcommand{\ConstantTok}[1]{\textcolor[rgb]{0.56,0.35,0.01}{#1}}
\newcommand{\ControlFlowTok}[1]{\textcolor[rgb]{0.00,0.23,0.31}{\textbf{#1}}}
\newcommand{\DataTypeTok}[1]{\textcolor[rgb]{0.68,0.00,0.00}{#1}}
\newcommand{\DecValTok}[1]{\textcolor[rgb]{0.68,0.00,0.00}{#1}}
\newcommand{\DocumentationTok}[1]{\textcolor[rgb]{0.37,0.37,0.37}{\textit{#1}}}
\newcommand{\ErrorTok}[1]{\textcolor[rgb]{0.68,0.00,0.00}{#1}}
\newcommand{\ExtensionTok}[1]{\textcolor[rgb]{0.00,0.23,0.31}{#1}}
\newcommand{\FloatTok}[1]{\textcolor[rgb]{0.68,0.00,0.00}{#1}}
\newcommand{\FunctionTok}[1]{\textcolor[rgb]{0.28,0.35,0.67}{#1}}
\newcommand{\ImportTok}[1]{\textcolor[rgb]{0.00,0.46,0.62}{#1}}
\newcommand{\InformationTok}[1]{\textcolor[rgb]{0.37,0.37,0.37}{#1}}
\newcommand{\KeywordTok}[1]{\textcolor[rgb]{0.00,0.23,0.31}{\textbf{#1}}}
\newcommand{\NormalTok}[1]{\textcolor[rgb]{0.00,0.23,0.31}{#1}}
\newcommand{\OperatorTok}[1]{\textcolor[rgb]{0.37,0.37,0.37}{#1}}
\newcommand{\OtherTok}[1]{\textcolor[rgb]{0.00,0.23,0.31}{#1}}
\newcommand{\PreprocessorTok}[1]{\textcolor[rgb]{0.68,0.00,0.00}{#1}}
\newcommand{\RegionMarkerTok}[1]{\textcolor[rgb]{0.00,0.23,0.31}{#1}}
\newcommand{\SpecialCharTok}[1]{\textcolor[rgb]{0.37,0.37,0.37}{#1}}
\newcommand{\SpecialStringTok}[1]{\textcolor[rgb]{0.13,0.47,0.30}{#1}}
\newcommand{\StringTok}[1]{\textcolor[rgb]{0.13,0.47,0.30}{#1}}
\newcommand{\VariableTok}[1]{\textcolor[rgb]{0.07,0.07,0.07}{#1}}
\newcommand{\VerbatimStringTok}[1]{\textcolor[rgb]{0.13,0.47,0.30}{#1}}
\newcommand{\WarningTok}[1]{\textcolor[rgb]{0.37,0.37,0.37}{\textit{#1}}}

\providecommand{\tightlist}{%
  \setlength{\itemsep}{0pt}\setlength{\parskip}{0pt}}\usepackage{longtable,booktabs,array}
\usepackage{calc} % for calculating minipage widths
% Correct order of tables after \paragraph or \subparagraph
\usepackage{etoolbox}
\makeatletter
\patchcmd\longtable{\par}{\if@noskipsec\mbox{}\fi\par}{}{}
\makeatother
% Allow footnotes in longtable head/foot
\IfFileExists{footnotehyper.sty}{\usepackage{footnotehyper}}{\usepackage{footnote}}
\makesavenoteenv{longtable}
\usepackage{graphicx}
\makeatletter
\newsavebox\pandoc@box
\newcommand*\pandocbounded[1]{% scales image to fit in text height/width
  \sbox\pandoc@box{#1}%
  \Gscale@div\@tempa{\textheight}{\dimexpr\ht\pandoc@box+\dp\pandoc@box\relax}%
  \Gscale@div\@tempb{\linewidth}{\wd\pandoc@box}%
  \ifdim\@tempb\p@<\@tempa\p@\let\@tempa\@tempb\fi% select the smaller of both
  \ifdim\@tempa\p@<\p@\scalebox{\@tempa}{\usebox\pandoc@box}%
  \else\usebox{\pandoc@box}%
  \fi%
}
% Set default figure placement to htbp
\def\fps@figure{htbp}
\makeatother

\makeatletter
\@ifpackageloaded{caption}{}{\usepackage{caption}}
\AtBeginDocument{%
\ifdefined\contentsname
  \renewcommand*\contentsname{Table of contents}
\else
  \newcommand\contentsname{Table of contents}
\fi
\ifdefined\listfigurename
  \renewcommand*\listfigurename{List of Figures}
\else
  \newcommand\listfigurename{List of Figures}
\fi
\ifdefined\listtablename
  \renewcommand*\listtablename{List of Tables}
\else
  \newcommand\listtablename{List of Tables}
\fi
\ifdefined\figurename
  \renewcommand*\figurename{Figure}
\else
  \newcommand\figurename{Figure}
\fi
\ifdefined\tablename
  \renewcommand*\tablename{Table}
\else
  \newcommand\tablename{Table}
\fi
}
\@ifpackageloaded{float}{}{\usepackage{float}}
\floatstyle{ruled}
\@ifundefined{c@chapter}{\newfloat{codelisting}{h}{lop}}{\newfloat{codelisting}{h}{lop}[chapter]}
\floatname{codelisting}{Listing}
\newcommand*\listoflistings{\listof{codelisting}{List of Listings}}
\makeatother
\makeatletter
\makeatother
\makeatletter
\@ifpackageloaded{caption}{}{\usepackage{caption}}
\@ifpackageloaded{subcaption}{}{\usepackage{subcaption}}
\makeatother
\journal{Journal Name}

\usepackage[]{natbib}
\bibliographystyle{elsarticle-harv}
\usepackage{bookmark}

\IfFileExists{xurl.sty}{\usepackage{xurl}}{} % add URL line breaks if available
\urlstyle{same} % disable monospaced font for URLs
\hypersetup{
  pdftitle={A tutorial of Bayesian beta regressions with brms in R},
  pdfauthor={Stefano Coretta; Paul Bürkner},
  pdfkeywords={keyword1, keyword2},
  colorlinks=true,
  linkcolor={blue},
  filecolor={Maroon},
  citecolor={Blue},
  urlcolor={Blue},
  pdfcreator={LaTeX via pandoc}}


\setlength{\parindent}{6pt}
\begin{document}

\begin{frontmatter}
\title{A tutorial of Bayesian beta regressions with brms in R}
\author[1]{Stefano Coretta%
\corref{cor1}%
\fnref{fn1}}
 \ead{s.coretta@ed.ac.uk} 
\author[]{Paul Bürkner%
%
}
 \ead{paul.buerkner@gmail.com} 

\affiliation[1]{organization={University of Edinburgh, Linguistics and
English Language},addressline={3 George
Sq},city={Edinburgh},postcode={EH8 9AD},postcodesep={}}

\cortext[cor1]{Corresponding author}
\fntext[fn1]{This is the first author footnote.}

        
\begin{abstract}
This is the abstract. Lorem ipsum dolor sit amet, consectetur adipiscing
elit. Vestibulum augue turpis, dictum non malesuada a, volutpat eget
velit. Nam placerat turpis purus, eu tristique ex tincidunt et. Mauris
sed augue eget turpis ultrices tincidunt. Sed et mi in leo porta
egestas. Aliquam non laoreet velit. Nunc quis ex vitae eros aliquet
auctor nec ac libero. Duis laoreet sapien eu mi luctus, in bibendum leo
molestie. Sed hendrerit diam diam, ac dapibus nisl volutpat vitae.
Aliquam bibendum varius libero, eu efficitur justo rutrum at. Sed at
tempus elit.
\end{abstract}





\begin{keyword}
    keyword1 \sep 
    keyword2
\end{keyword}
\end{frontmatter}
    

\section{Introduction}\label{introduction}

Phonetic research often involves numeric continuous outcome variables,
like durations, frequencies, loudness and ratios. Another commonly
employed type of outcome variable are proportions: for example,
proportion of voicing during closure, vocal folds contact quotient,
gesture amplitude, nasalance. Moreover, virtually any measure can be
MIN-MAX normalised, a procedure which transforms values so that they are
in the range 0--1.

Regression models are very common, but there is a tendency of using
Gaussian distribution families (i.e.~probability distributions for the
outcome variable) for anything that is numeric. A possible reason is
that the base R function for fitting regression models, \texttt{lm()},
and the lme4 function used to fit regression models with varying terms,
\texttt{lmer()}, both fit Gaussian regressions by default and the user
does not have to specify the distribution family. This tacit defaulting
to Gaussian models is also reflected in teaching practices, where test
and models using the Gaussian distribution are the first to be taught,
due to their relative simplicity and the fact that other models are
generalisations of Gaussian models.

However, proportion are naturally not Gaussian, since they are limited
between 0 and 1. The theoretical distribution that generates values with
this characteristics is the beta distribution. Thus, regression models
with proportions as the outcome variable should be fitted using a beta
distribution as the distribution family. This tutorial introduces
researchers to beta regression models in R using the package brms.
Familiarity with regression modelling in R with a package like lme4 is
assumed, but no prior knowledge of Bayesian statistics is necessary.

\section{The beta distribution}\label{the-beta-distribution}

\ldots{}

\section{Case study 1: voicing within consonant
closure}\label{case-study-1-voicing-within-consonant-closure}

For the first case study, we will model the proportion of voicing within
consonant closure. The measurements come from a data set of audio and
electroglottographic (EGG) recordings of 19 speakers of Northwestern
Italian. The participants read frame sentences which included target
words of the form /CVCo/, where /C/ was either /k, t, p/ in all
permutations and /V/ was either /i, e, a, ɔ, u/ (two resulting words,
/peto/ and /kako/ were excluded because they are profanities), for a
total of 43 target words. There were 4 different frame sentence, with a
total of 172 trials per participant (3,268 grand total). The actual
observation count is 2,419, after removing speech errors, EGG
measurement errors, and cases in which voicing ceased before the closure
onset/after the closure offset of the second /C/.

The proportion of voicing during the closure of the second /C/ was
calculated as the proportion of contiguous voicing duration after
closure onset to total duration of closure. The following code chunk
attaches the tidyverse packages (for reading and wrangling data) and
loads the \texttt{ita\_egg} tibble (data frame). The tibble is filtered
so as to remove voicing proportions (\texttt{voi\_clo\_prop}) that are
smaller than 0 and greater than 1. The variables \texttt{vowel} and
\texttt{c2} are converted to factors to specify the order of the levels.

\begin{Shaded}
\begin{Highlighting}[]
\CommentTok{\# attach tidyverse and set light theme for plots}
\FunctionTok{library}\NormalTok{(tidyverse)}
\FunctionTok{theme\_set}\NormalTok{(}\FunctionTok{theme\_light}\NormalTok{())}

\CommentTok{\# load tibble}
\FunctionTok{load}\NormalTok{(}\StringTok{"data/coretta2018/ita\_egg.rda"}\NormalTok{)}

\CommentTok{\# filter and mutate data}
\NormalTok{ita\_egg }\OtherTok{\textless{}{-}}\NormalTok{ ita\_egg }\SpecialCharTok{|\textgreater{}} 
  \FunctionTok{filter}\NormalTok{(voi\_clo\_prop }\SpecialCharTok{\textgreater{}} \DecValTok{0}\NormalTok{, voi\_clo\_prop }\SpecialCharTok{\textless{}} \DecValTok{1}\NormalTok{) }\SpecialCharTok{|\textgreater{}} 
  \FunctionTok{mutate}\NormalTok{(}
    \AttributeTok{vowel =} \FunctionTok{factor}\NormalTok{(vowel, }\AttributeTok{levels =} \FunctionTok{c}\NormalTok{(}\StringTok{"i"}\NormalTok{, }\StringTok{"e"}\NormalTok{, }\StringTok{"a"}\NormalTok{, }\StringTok{"o"}\NormalTok{, }\StringTok{"u"}\NormalTok{)),}
    \AttributeTok{c2 =} \FunctionTok{factor}\NormalTok{(c2, }\AttributeTok{levels =} \FunctionTok{c}\NormalTok{(}\StringTok{"k"}\NormalTok{, }\StringTok{"t"}\NormalTok{, }\StringTok{"p"}\NormalTok{))}
\NormalTok{  )}
\end{Highlighting}
\end{Shaded}

Here is what the tibble looks like.

\begin{Shaded}
\begin{Highlighting}[]
\NormalTok{ita\_egg}
\end{Highlighting}
\end{Shaded}

\begin{verbatim}
# A tibble: 2,419 x 53
   speaker ipu    stimulus    sentence_ons sentence_off word_ons word_off v1_ons
   <chr>   <chr>  <chr>              <dbl>        <dbl>    <dbl>    <dbl>  <dbl>
 1 it01    ipu_1  Ripete 'po~         13.2         14.9     13.7     14.1   13.9
 2 it01    ipu_2  Ripete 'to~         16.9         18.6     17.4     17.9   17.5
 3 it01    ipu_3  Ripete 'pa~         20.2         21.9     20.7     21.1   20.8
 4 it01    ipu_4  Sentivo 't~         23.5         25.1     24.0     24.4   24.1
 5 it01    ipu_5  Sentivo 't~         26.3         27.8     26.8     27.2   26.9
 6 it01    ipu_6  Scrivete '~         29.2         30.9     29.7     30.1   29.8
 7 it01    ipu_7  Sentivo 'c~         32.1         33.6     32.6     33.1   32.8
 8 it01    ipu_8  Scrivete '~         35.0         36.6     35.5     35.9   35.6
 9 it01    ipu_9  Ha detto '~         41.9         43.5     42.3     42.7   42.4
10 it01    ipu_10 Sentivo 'p~         47.4         48.9     47.9     48.4   48.1
# i 2,409 more rows
# i 45 more variables: c2_ons <dbl>, v2_ons <dbl>, voice_ons <dbl>,
#   voice_off <dbl>, c1_rel <dbl>, c2_rel <dbl>, stimulus_id <dbl>,
#   sentence <chr>, word <chr>, c1 <chr>, vowel <fct>, c2 <fct>,
#   backness <chr>, height <fct>, c1_place <fct>, c2_place <fct>,
#   v1_duration <dbl>, c2_clos_duration <dbl>, rel_voff <dbl>,
#   sent_duration <dbl>, speech_rate <dbl>, speech_rate_c <dbl>, ...
\end{verbatim}

Figure~\ref{fig-ita-egg} shows the raw voicing duration proportion
values split by vowel /i, e, a, ɔ, u/ and second consonant /k, t, p/ in
the /CVCo/ target words. The plot suggests that, on average, the voicing
proportion is slightly lower with /k/ than with /t, p/. Moreover, there
is greater variability between vowels in /t, p/ than in /k/. We will use
a beta regression model to assess these patterns. {[}``expectations''
XXX{]}

\begin{figure}

\centering{

\pandocbounded{\includegraphics[keepaspectratio]{manuscript_files/figure-pdf/fig-ita-egg-1.pdf}}

}

\caption{\label{fig-ita-egg}Proportion of voicing during the closure of
the second consonant in /CVCo/ words by vowel and the second consonant.}

\end{figure}%

We will use brms to fit Bayesian beta regressions. {[}XXX why
Bayesian{]}. The model has voicing proportion as the outcome variable
and the following terms: an interaction between vowel (/i, e, a, ɔ, u/)
and second consonant C2 (/k, t, p/), centred speech rate (number of
syllables per second); as varying (aka random) terms, by-speaker varying
coefficients for the vowel/consonant interaction and for centred speech
rate.\footnote{Footnote about Gelman's terminology for random effects.}
The categorical predictors vowel and C2 are coded using indexing rather
than traditional R contrasts: in R, this corresponds to suppressing the
model's intercept with the \texttt{0\ +} syntax; using indexing instead
of contrasts makes it easier to specify priors. For pedagogical
simplicity, the model will use the default priors, but note that in real
data analyses contexts, priors should be specified by the user. I refer
the readers to XXX.

\begin{Shaded}
\begin{Highlighting}[]
\CommentTok{\# attach brms}
\FunctionTok{library}\NormalTok{(brms)}

\CommentTok{\# fit the model}
\CommentTok{\# Takes 3 minutes on MacBook Pro 2020, M1}
\NormalTok{voi\_prop\_bm }\OtherTok{\textless{}{-}} \FunctionTok{brm}\NormalTok{(}
  \CommentTok{\# model formula}
\NormalTok{  voi\_clo\_prop }\SpecialCharTok{\textasciitilde{}}
    \CommentTok{\# constant terms}
    \DecValTok{0} \SpecialCharTok{+}\NormalTok{ vowel}\SpecialCharTok{:}\NormalTok{c2 }\SpecialCharTok{+}\NormalTok{ speech\_rate\_c }\SpecialCharTok{+}
    \CommentTok{\# varying terms}
\NormalTok{    (}\DecValTok{0} \SpecialCharTok{+}\NormalTok{ vowel}\SpecialCharTok{:}\NormalTok{c2 }\SpecialCharTok{+}\NormalTok{ speech\_rate\_c }\SpecialCharTok{|}\NormalTok{ speaker),}
  \CommentTok{\# uses the beta family for the outcome}
  \AttributeTok{family =}\NormalTok{ Beta,}
  \AttributeTok{data =}\NormalTok{ ita\_egg,}
  \AttributeTok{cores =} \DecValTok{4}\NormalTok{,}
  \AttributeTok{seed =} \DecValTok{3749}\NormalTok{,}
  \AttributeTok{file =} \StringTok{"data/cache/voi\_prop\_bm"}
\NormalTok{)}
\end{Highlighting}
\end{Shaded}

The \texttt{summary()} function prints the full model summary. For
conciseness, we will use the \texttt{fixef()} function which prints the
regression coefficients. {[}EXPLAIN EXPECTED PREDICTIONS XXX{]}. The
full summary with an explanation of each part can be found in XXX. For
each coefficient in the model, \texttt{fixef()} prints the name of the
coefficient, the mean estimate, the estimate error and the lower and
upper limits of a Bayesian Credible interval (CrI). Here, we print an
80\% CrI. There is nothing special about 95\% CrI within Bayesian
inference and instead experts recommend to check and report a variety of
CrIs. Bayesian CrIs indicate that at a certain probability levels the
``true'' estimate lies within that interval: so, for example, a 90\% CrI
{[}A, B{]} indicates that there is a 90\% probability that the ``true''
estimate is between A and B. Different probability levels correspond to
different levels of confidence: the higher the probability the higher
the confidence (always conditional on data and model). Obtaining CrIs at
different probability levels allows researchers to make more
fine-grained inferential statements than the frequentist significance
dichotomy affords. For simplicity of exposition, we will use 80\% CrIs
in this case study but we strongly recommend researchers to always
obtain CrIs at different levels of probability and base their inferences
on all and not one in particular. To reiterate, in Bayesian inference,
an 80\% CrI indicates the range of values within which the true estimate
falls at 80\% probability or confidence. We round all values to the
nearest 2 digits for clarity.

\begin{Shaded}
\begin{Highlighting}[]
\FunctionTok{round}\NormalTok{(}
  \FunctionTok{fixef}\NormalTok{(voi\_prop\_bm, }\AttributeTok{prob =} \FunctionTok{c}\NormalTok{(}\FloatTok{0.1}\NormalTok{, }\FloatTok{0.9}\NormalTok{)),}
  \AttributeTok{digits =} \DecValTok{2}
\NormalTok{)}
\end{Highlighting}
\end{Shaded}

\begin{verbatim}
              Estimate Est.Error   Q10   Q90
speech_rate_c     0.08      0.06  0.01  0.15
voweli:c2k       -0.91      0.14 -1.08 -0.74
vowele:c2k       -1.08      0.11 -1.22 -0.94
vowela:c2k       -0.99      0.12 -1.14 -0.84
vowelo:c2k       -0.79      0.14 -0.96 -0.62
vowelu:c2k       -1.00      0.16 -1.20 -0.80
voweli:c2t       -0.66      0.11 -0.79 -0.53
vowele:c2t       -0.84      0.14 -1.02 -0.66
vowela:c2t       -1.43      0.13 -1.60 -1.26
vowelo:c2t       -1.15      0.13 -1.31 -0.99
vowelu:c2t       -0.68      0.12 -0.83 -0.54
voweli:c2p       -0.68      0.11 -0.81 -0.54
vowele:c2p       -0.88      0.15 -1.07 -0.68
vowela:c2p       -1.14      0.13 -1.31 -0.98
vowelo:c2p       -0.66      0.11 -0.80 -0.53
vowelu:c2p       -0.44      0.12 -0.59 -0.28
\end{verbatim}

The coefficients of a beta regression are estimated on the log-odds
scale, as in Bernoulli/binomial (aka logistic) regressions. From the
summary, we see that speech rate (number of syllables per second) has a
positive effect on voicing proportion: the 80\% CrI is between 0.01 and
0.15 log-odds {[}\(\beta\) = 0.08, SD = 0.06{]}. Log-odds can be
converted to odd-ratios by exponentiating the value: 0.01-0.15 log odds
correspond to an odd-ratio of 1.01 to 1.16, or as percentages, to an
increase of voicing of 1 to 16\% for every increase of one syllable per
second. Since this is an 80\% CrI, we can be 80\% confident that the
true effect of speech rate is between 1-16\% increase of voicing
proportion, conditional on the data and model. Note that transforming
measures this way is appropriate \emph{only} with quantile-based
measures (like CrIs) but not with moments like the mean and standard
deviation: to correctly get mean and SDs in the transformed scale, you
must first extract the posterior draws (see below), convert them and
then take moments such as mean and SD (for a more detailed explanation,
see XXX). In the avoidance of doubt, we will always transform the drawn
values first and then take summary measures.

Turning now to the coefficients for vowel and C2, given the indexing
approach of coding these variables the model summary and the output of
\texttt{fixef()} reports the \emph{predictions} in log-odds for each
combination of vowel and C2, rather than differences between levels. The
CrIs of the vowel/C2 coefficients span all negative log-odds values:
these correspond to proportions that are lower than 0.5 (which is 0 in
log-odds). This matches the general trends in the raw data, which we
plotted in Figure~\ref{fig-ita-egg}.

Next, we will plot the predicted proportions of each vowel/C2
combination at mean speech rate (i.e.~centred speech rate = 0) and then
calculate the average pair-wise difference in voicing proportion between
/k, t, p/. Finally, we will assess whether there is greater
between-vowel variability in /t, p/ relative to /k/.

Before being able to plot the predictions, it's important to get
familiar with the so-called posterior draws. {[}Bayesian MCMC XXX{]}.
Posterior draws can be conveniently obtained with the
\texttt{as\_draws\_df()}. For the moment, we will extract only the draws
of the constant regression coefficients (model variables starting with
\texttt{b\_}). To check which coefficients are available in a model, use
\texttt{get\_variables()} from the tidybayes package.
\texttt{as\_draws\_df()} returns a tibble where each column contains the
drawn values of a coefficient. The probability distribution of the drawn
values of each coefficient is the posterior probability distribution of
that coefficient. Note that, due to the indexing coding of vowel and C2,
all coefficient except \texttt{b\_speech\_rate\_c} are \emph{predicted
log-odds} for each vowel/C2 combination (the drawn values for
\texttt{b\_speech\_rate\_c} are drawn \emph{differences} in log-odds for
each unit increase of speech rate). The drawn values are in log-odds,
but we can convert them to proportions with \texttt{plogis()} (we will
do this when plotting below).

\begin{Shaded}
\begin{Highlighting}[]
\CommentTok{\# extract only coefficient variables starting with "b\_"}
\NormalTok{voi\_prop\_bm\_draws }\OtherTok{\textless{}{-}} \FunctionTok{as\_draws\_df}\NormalTok{(voi\_prop\_bm, }\AttributeTok{variable =} \StringTok{"\^{}b\_"}\NormalTok{, }\AttributeTok{regex =} \ConstantTok{TRUE}\NormalTok{)}
\NormalTok{voi\_prop\_bm\_draws}
\end{Highlighting}
\end{Shaded}

\begin{verbatim}
# A draws_df: 1000 iterations, 4 chains, and 16 variables
   b_speech_rate_c b_voweli:c2k b_vowele:c2k b_vowela:c2k b_vowelo:c2k
1            0.159        -0.72        -1.05        -1.12        -1.02
2            0.035        -0.75        -1.39        -1.06        -0.89
3            0.126        -0.85        -0.98        -1.10        -0.64
4            0.064        -0.70        -1.07        -1.05        -0.83
5            0.032        -0.77        -1.18        -0.94        -0.70
6            0.073        -0.85        -1.23        -1.06        -0.77
7            0.101        -0.95        -1.07        -1.14        -0.86
8            0.074        -1.02        -1.07        -1.17        -0.83
9            0.087        -1.04        -1.07        -1.26        -0.79
10           0.092        -1.20        -1.04        -0.92        -0.64
   b_vowelu:c2k b_voweli:c2t b_vowele:c2t
1         -1.05        -0.62        -0.80
2         -1.04        -0.67        -0.99
3         -0.91        -0.66        -0.80
4         -1.07        -0.53        -0.87
5         -1.06        -0.63        -0.87
6         -1.00        -0.69        -0.73
7         -1.22        -0.74        -1.08
8         -1.22        -0.73        -1.04
9         -1.14        -0.66        -0.91
10        -0.84        -0.63        -0.61
# ... with 3990 more draws, and 8 more variables
# ... hidden reserved variables {'.chain', '.iteration', '.draw'}
\end{verbatim}

We can now wrangle this tibble and plot the posterior distributions for
each vowel/C2 combination.

\begin{Shaded}
\begin{Highlighting}[]
\NormalTok{voi\_prop\_bm\_draws\_long }\OtherTok{\textless{}{-}}\NormalTok{ voi\_prop\_bm\_draws }\SpecialCharTok{|\textgreater{}} 
  \CommentTok{\# drop b\_speech\_rate\_c before pivoting}
  \FunctionTok{select}\NormalTok{(}\SpecialCharTok{{-}}\NormalTok{b\_speech\_rate\_c) }\SpecialCharTok{|\textgreater{}} 
  \CommentTok{\# pivot vowel:c2 columns}
  \FunctionTok{pivot\_longer}\NormalTok{(}\StringTok{\textasciigrave{}}\AttributeTok{b\_voweli:c2k}\StringTok{\textasciigrave{}}\SpecialCharTok{:}\StringTok{\textasciigrave{}}\AttributeTok{b\_vowelu:c2p}\StringTok{\textasciigrave{}}\NormalTok{, }\AttributeTok{names\_to =} \StringTok{"coeff"}\NormalTok{) }\SpecialCharTok{|\textgreater{}} 
  \CommentTok{\# separate "coeff" labels into type ("b"), vowel and c2}
  \FunctionTok{separate}\NormalTok{(coeff, }\AttributeTok{into =} \FunctionTok{c}\NormalTok{(}\StringTok{"type"}\NormalTok{, }\StringTok{"vowel"}\NormalTok{, }\StringTok{"c2"}\NormalTok{))}
\end{Highlighting}
\end{Shaded}

\begin{verbatim}
Warning: Dropping 'draws_df' class as required metadata was removed.
\end{verbatim}

\begin{Shaded}
\begin{Highlighting}[]
\NormalTok{voi\_prop\_bm\_draws\_long}
\end{Highlighting}
\end{Shaded}

\begin{verbatim}
# A tibble: 60,000 x 7
   .chain .iteration .draw type  vowel  c2     value
    <int>      <int> <int> <chr> <chr>  <chr>  <dbl>
 1      1          1     1 b     voweli c2k   -0.721
 2      1          1     1 b     vowele c2k   -1.05 
 3      1          1     1 b     vowela c2k   -1.12 
 4      1          1     1 b     vowelo c2k   -1.02 
 5      1          1     1 b     vowelu c2k   -1.05 
 6      1          1     1 b     voweli c2t   -0.625
 7      1          1     1 b     vowele c2t   -0.795
 8      1          1     1 b     vowela c2t   -1.40 
 9      1          1     1 b     vowelo c2t   -1.10 
10      1          1     1 b     vowelu c2t   -0.668
# i 59,990 more rows
\end{verbatim}

For plotting, we can use ggplot2 statistics layers from the ggdist
package. \texttt{stat\_halfeye()} plots the density of the posterior
probability (in grey), its median (point) and CrIs (lines). Let's use 60
and 80\% CrIs and transform the log-odds values to proportions with
\texttt{plogis()}.

\begin{Shaded}
\begin{Highlighting}[]
\CommentTok{\# attach ggdist package}
\FunctionTok{library}\NormalTok{(ggdist)}
\end{Highlighting}
\end{Shaded}

\begin{verbatim}

Attaching package: 'ggdist'
\end{verbatim}

\begin{verbatim}
The following objects are masked from 'package:brms':

    dstudent_t, pstudent_t, qstudent_t, rstudent_t
\end{verbatim}

\begin{Shaded}
\begin{Highlighting}[]
\NormalTok{voi\_prop\_bm\_draws\_long }\SpecialCharTok{|\textgreater{}} 
  \FunctionTok{ggplot}\NormalTok{(}\FunctionTok{aes}\NormalTok{(}\FunctionTok{plogis}\NormalTok{(value), vowel)) }\SpecialCharTok{+}
  \FunctionTok{stat\_halfeye}\NormalTok{(}\AttributeTok{.width =} \FunctionTok{c}\NormalTok{(}\FloatTok{0.6}\NormalTok{, }\FloatTok{0.8}\NormalTok{)) }\SpecialCharTok{+}
  \FunctionTok{facet\_grid}\NormalTok{(}\AttributeTok{rows =} \FunctionTok{vars}\NormalTok{(c2))}
\end{Highlighting}
\end{Shaded}

\begin{figure}[H]

\centering{

\pandocbounded{\includegraphics[keepaspectratio]{manuscript_files/figure-pdf/fig-voi-prop-bm-1-1.pdf}}

}

\caption{\label{fig-voi-prop-bm-1}}

\end{figure}%

What if we want to plot the average predicted voicing proportion for the
three consonants /k, t, p/? One approach is to take the mean across
vowels within each consonant for each posterior draw, and the posterior
distribution of the resulting list of values is the predicted posterior
distribution of voicing proportion for each consonant, assuming an
``average'' vowel.

\begin{Shaded}
\begin{Highlighting}[]
\NormalTok{voi\_prop\_bm\_draws\_long\_c2 }\OtherTok{\textless{}{-}}\NormalTok{ voi\_prop\_bm\_draws\_long }\SpecialCharTok{|\textgreater{}} 
  \CommentTok{\# grouping by .draw and c2 ensures that averaging applies only within draw and c2}
  \FunctionTok{group\_by}\NormalTok{(.draw, c2) }\SpecialCharTok{|\textgreater{}} 
  \CommentTok{\# calculate the mean value within draw/c2}
  \FunctionTok{summarise}\NormalTok{(}
    \AttributeTok{value\_mean =} \FunctionTok{mean}\NormalTok{(value), }\AttributeTok{.groups =} \StringTok{"drop"}
\NormalTok{  )}

\NormalTok{voi\_prop\_bm\_draws\_long\_c2 }\SpecialCharTok{|\textgreater{}} 
  \FunctionTok{ggplot}\NormalTok{(}\FunctionTok{aes}\NormalTok{(}\FunctionTok{plogis}\NormalTok{(value\_mean), c2)) }\SpecialCharTok{+}
  \FunctionTok{stat\_halfeye}\NormalTok{(}\AttributeTok{.width =} \FunctionTok{c}\NormalTok{(}\FloatTok{0.6}\NormalTok{, }\FloatTok{0.8}\NormalTok{))}
\end{Highlighting}
\end{Shaded}

\pandocbounded{\includegraphics[keepaspectratio]{manuscript_files/figure-pdf/voi-prop-bm-draws-long-c2-1.pdf}}

Based on the posterior distributions of the mean voicing proportion by
consonant, /p/ has a somewhat higher voicing proportion than /k/ and
/t/. The real question is: how much higher? We can quantify this by
taking the difference of the drawn values for /p/ and those for /t, k/.
Since we want to compare /t, k/ to /p/, we should first average the
draws of /t, k/ and then take the difference of the averaged draws and
the draws of /p/.

\begin{Shaded}
\begin{Highlighting}[]
\NormalTok{voi\_prop\_bm\_diff }\OtherTok{\textless{}{-}}\NormalTok{ voi\_prop\_bm\_draws\_long\_c2 }\SpecialCharTok{|\textgreater{}} 
  \CommentTok{\# pivot data to create one column per consonant with the mean drawn values,}
  \CommentTok{\# with one draw per raw}
  \FunctionTok{pivot\_wider}\NormalTok{(}\AttributeTok{names\_from =}\NormalTok{ c2, }\AttributeTok{values\_from =}\NormalTok{ value\_mean) }\SpecialCharTok{|\textgreater{}} 
  \FunctionTok{mutate}\NormalTok{(}
    \CommentTok{\# calculate the mean of /k/ and /t/, for each draw}
    \AttributeTok{c2tk =} \FunctionTok{mean}\NormalTok{(}\FunctionTok{c}\NormalTok{(c2k, c2t)),}
    \CommentTok{\# calculate the difference of /p/ and /t, k/}
    \AttributeTok{c2p\_tk\_diff =}\NormalTok{ c2p }\SpecialCharTok{{-}}\NormalTok{ c2tk}
\NormalTok{  )}

\NormalTok{voi\_prop\_bm\_diff}
\end{Highlighting}
\end{Shaded}

\begin{verbatim}
# A tibble: 4,000 x 6
   .draw    c2k    c2p    c2t   c2tk c2p_tk_diff
   <int>  <dbl>  <dbl>  <dbl>  <dbl>       <dbl>
 1     1 -0.992 -0.745 -0.917 -0.953      0.208 
 2     2 -1.02  -0.790 -1.03  -0.953      0.163 
 3     3 -0.895 -0.687 -0.890 -0.953      0.266 
 4     4 -0.944 -0.814 -0.910 -0.953      0.140 
 5     5 -0.928 -0.791 -0.920 -0.953      0.162 
 6     6 -0.983 -0.728 -0.926 -0.953      0.225 
 7     7 -1.05  -0.873 -1.08  -0.953      0.0802
 8     8 -1.06  -0.875 -1.07  -0.953      0.0782
 9     9 -1.06  -0.974 -1.05  -0.953     -0.0206
10    10 -0.929 -0.829 -0.953 -0.953      0.125 
# i 3,990 more rows
\end{verbatim}

\begin{Shaded}
\begin{Highlighting}[]
\NormalTok{voi\_prop\_bm\_diff }\SpecialCharTok{|\textgreater{}} 
  \FunctionTok{ggplot}\NormalTok{(}\FunctionTok{aes}\NormalTok{(c2p\_tk\_diff)) }\SpecialCharTok{+}
  \FunctionTok{stat\_halfeye}\NormalTok{(}\AttributeTok{.width =} \FunctionTok{c}\NormalTok{(}\FloatTok{0.6}\NormalTok{, }\FloatTok{0.8}\NormalTok{, }\FloatTok{0.9}\NormalTok{)) }\SpecialCharTok{+}
  \FunctionTok{geom\_vline}\NormalTok{(}\AttributeTok{xintercept =} \DecValTok{0}\NormalTok{)}
\end{Highlighting}
\end{Shaded}

\begin{figure}[H]

\centering{

\pandocbounded{\includegraphics[keepaspectratio]{manuscript_files/figure-pdf/fig-diff-p-tk-1.pdf}}

}

\caption{\label{fig-diff-p-tk}}

\end{figure}%

Once we have the posterior difference, we can obtain CrIs of the
difference using \texttt{quantile2()} from the posterior package. Beware
that the values of the difference are in log-odds! We can convert these
into odd-ratios with \texttt{exp()}. odd-ratios indicate the ratio of
the difference between A and B, so that 1 means no difference, values
greater than 1 indicate an increase in A relative to B and values lower
than 1 indicate a decrease in A relative to B. odd-ratios are useful
when looking at differences that are in log-odds because while the
relative magnitude of the difference in proportion between two groups is
the same independent of the baseline proportion, the \emph{absolute}
magnitude of the difference depends on the baseline value. For example,
an odd-ratio difference of 1.25 would correspond to a proportion
increase of 13 percentage points if the baseline proportion is 0.62 but
it would correspond to a proportion increase of 17 percentage points if
the baseline proportion is 0.73. Of course, in real research contexts it
is still usefull to think about absolute magnitudes and their relevance
from a conceptual and methodological perspective. In this tutorial we
just focus on odd-ratios for simplicity.

\begin{Shaded}
\begin{Highlighting}[]
\FunctionTok{library}\NormalTok{(posterior)}
\end{Highlighting}
\end{Shaded}

\begin{verbatim}
This is posterior version 1.6.0
\end{verbatim}

\begin{verbatim}

Attaching package: 'posterior'
\end{verbatim}

\begin{verbatim}
The following objects are masked from 'package:stats':

    mad, sd, var
\end{verbatim}

\begin{verbatim}
The following objects are masked from 'package:base':

    %in%, match
\end{verbatim}

\begin{Shaded}
\begin{Highlighting}[]
\NormalTok{voi\_prop\_bm\_diff }\SpecialCharTok{|\textgreater{}} 
  \FunctionTok{mutate}\NormalTok{(}\AttributeTok{c2p\_tk\_diff\_ratio =} \FunctionTok{exp}\NormalTok{(c2p\_tk\_diff)) }\SpecialCharTok{|\textgreater{}} 
  \FunctionTok{reframe}\NormalTok{(}
    \CommentTok{\# 90\% CrI}
    \AttributeTok{q90 =} \FunctionTok{quantile2}\NormalTok{(c2p\_tk\_diff\_ratio, }\AttributeTok{probs =} \FunctionTok{c}\NormalTok{(}\FloatTok{0.05}\NormalTok{, }\FloatTok{0.95}\NormalTok{)),}
    \CommentTok{\# 80\% CrI}
    \AttributeTok{q80 =} \FunctionTok{quantile2}\NormalTok{(c2p\_tk\_diff\_ratio, }\AttributeTok{probs =} \FunctionTok{c}\NormalTok{(}\FloatTok{0.1}\NormalTok{, }\FloatTok{0.9}\NormalTok{)),}
    \CommentTok{\# 60\% CrI}
    \AttributeTok{q60 =} \FunctionTok{quantile2}\NormalTok{(c2p\_tk\_diff\_ratio, }\AttributeTok{probs =} \FunctionTok{c}\NormalTok{(}\FloatTok{0.2}\NormalTok{, }\FloatTok{0.8}\NormalTok{)),}
\NormalTok{  ) }\SpecialCharTok{|\textgreater{}} 
  \CommentTok{\# round to 2 digits}
  \FunctionTok{mutate}\NormalTok{(}\FunctionTok{across}\NormalTok{(}\FunctionTok{everything}\NormalTok{(), }\SpecialCharTok{\textasciitilde{}}\FunctionTok{round}\NormalTok{(.x, }\DecValTok{2}\NormalTok{)))}
\end{Highlighting}
\end{Shaded}

\begin{verbatim}
# A tibble: 2 x 3
    q90   q80   q60
  <dbl> <dbl> <dbl>
1  1.07  1.1   1.14
2  1.38  1.34  1.3 
\end{verbatim}

So, based on the model and data, there is a 90\% probability that the
voicing proportion in /p/ is 1.07-1.38 times longer (or 7-38\% increase)
than in /t, k/. At 80\% confidence, the change ratio is 1.10-1.34 (or
10-34\% increase) while at 60\% confidence is 1.14-1.30 (14-30\%
increase). In other words we can be quite confident that the voicing
proportion in /p/ is longer than in /t, k/ and that the increase is less
than 35\%.

brms comes with a convenient function, \texttt{conditional\_effects()},
to plot posterior means and CrIs based on predictors in the model. In
the following example, we plot the predicted proportion of voicing by
consonant and vowel.

\begin{Shaded}
\begin{Highlighting}[]
\FunctionTok{conditional\_effects}\NormalTok{(voi\_prop\_bm, }\StringTok{"c2:vowel"}\NormalTok{)}
\end{Highlighting}
\end{Shaded}

\pandocbounded{\includegraphics[keepaspectratio]{manuscript_files/figure-pdf/voi-prop-bm-cond-1.pdf}}

Finally, the package marginaleffects {[}XXX{]} has two other convenience
functions that return CrIs of comparisons across predictor levels
(\texttt{avg\_comparisons()}) and CrIs of posterior predictions across
predictor levels (\texttt{avg\_predictions}). {[}XXX{]}

\begin{Shaded}
\begin{Highlighting}[]
\FunctionTok{library}\NormalTok{(marginaleffects)}

\FunctionTok{avg\_comparisons}\NormalTok{(voi\_prop\_bm, }\AttributeTok{variables =} \FunctionTok{list}\NormalTok{(}\AttributeTok{c2 =} \StringTok{"pairwise"}\NormalTok{), }\AttributeTok{conf\_level =} \FloatTok{0.8}\NormalTok{, }\AttributeTok{type =} \StringTok{"link"}\NormalTok{)}
\end{Highlighting}
\end{Shaded}

\begin{verbatim}

          Contrast Estimate   10.0 % 90.0 %
 mean(t) - mean(k)   0.0348 -0.00418 0.0738
 mean(p) - mean(k)   0.2176  0.17837 0.2558
 mean(p) - mean(t)   0.1829  0.14474 0.2192

Term: c2
Type:  link 
Columns: term, contrast, estimate, conf.low, conf.high, predicted_lo, predicted_hi, predicted, tmp_idx 
\end{verbatim}

\begin{Shaded}
\begin{Highlighting}[]
\FunctionTok{avg\_predictions}\NormalTok{(voi\_prop\_bm, }\AttributeTok{variables =} \StringTok{"vowel"}\NormalTok{, }\AttributeTok{conf\_level =} \FloatTok{0.8}\NormalTok{)}
\end{Highlighting}
\end{Shaded}

\begin{verbatim}

 vowel Estimate 10.0 % 90.0 %
     i    0.331  0.325  0.338
     e    0.300  0.293  0.307
     a    0.245  0.238  0.252
     o    0.305  0.298  0.312
     u    0.348  0.341  0.354

Type:  response 
Columns: vowel, estimate, conf.low, conf.high 
\end{verbatim}

\section{Case study 2: coarticulatory vowel
nasalisation}\label{case-study-2-coarticulatory-vowel-nasalisation}

For the second case study we will use data from \citet{carignan2021}.
The study looked at properties of nasality in German VNC sequences.
Here, we will focus on the effect of C voicing (voiceless /t/ vs voiced
/d/) on the proportion of nasalisation within the vowel in the VNC
sequence. Previous work on coarticulatory nasalisation in English has
suggested that vowels followed by an NC sequence where C is voiceless
(NT) should show earlier coarticulatory nasalisation than vowels
followed by an NC sequence where C is voiced \citep[ND, see review
in][]{carignan2021}. This pattern has been suggested to be driven by the
articulatory and acoustic incompatibility of voicelessness and
nasalisation, by which the velum opening gesture of the nasal consonant
is pushed away (i.e.~earlier) when the consonant following the nasal is
voiceless. Everything else being equal, a greater proportion of vowel
nasalisation (from the perspective of time) should be found in vowels
followed by NT than in vowels followed by ND.

We will model the proportion of coarticulatory nasalisation in the
German short vowels /i, e, a, o, u/ when followed by /nt/ or /nd/, using
a Bayesian beta regression model. The proportion was calculated as the
proportion of the nasal interval to the duration of the vowel. The nasal
interval was defined thus: the interval between the time of peak
velocity of velum opening to the offset of the vowel. We will use the
results of the regression model to answer the following questions:

\begin{enumerate}
\def\labelenumi{\arabic{enumi}.}
\tightlist
\item
  Is the nasalisation proportion, on average across vowels, greater in
  voiceless NC sequences?
\item
  Is there individual speaker variation?
\end{enumerate}

First, let's read and plot the data.

\begin{Shaded}
\begin{Highlighting}[]
\NormalTok{nasal }\OtherTok{\textless{}{-}} \FunctionTok{read\_csv}\NormalTok{(}\StringTok{"data/carignan2021/nasal.csv"}\NormalTok{)}
\NormalTok{nasal}
\end{Highlighting}
\end{Shaded}

\begin{verbatim}
# A tibble: 705 x 6
   speaker label               vowel NC    voicing   nas_prop
   <chr>   <chr>               <chr> <chr> <chr>        <dbl>
 1 S03     b__U_nt_@___N_B17/s u     nt    voiceless  0.367  
 2 S03     b__a_nd_@___N_B19/s a     nd    voiced     0.195  
 3 S03     b__a_nt_@___N_B15/s a     nt    voiceless  0.279  
 4 S03     f__I_nt_@___N_B05/s i     nt    voiceless  0.764  
 5 S03     l__I_nd_@___N_B06/s i     nd    voiced     0.00529
 6 S03     p__E_nt_____N_B09/s e     nt    voiceless  0.335  
 7 S03     r__a_nt_@___N_B06/s a     nt    voiceless  0.243  
 8 S03     v__I_nd_@___N_B07/s i     nd    voiced     0.0248 
 9 S03     v__I_nt_6___N_B15/s i     nt    voiceless  0.135  
10 S03     z__E_nd_@___N_B17/s e     nd    voiced     0.538  
# i 695 more rows
\end{verbatim}

\begin{itemize}
\item
  \texttt{speaker} indicates the speaker ID.
\item
  \texttt{label} is the word label as given in the original data.
\item
  \texttt{vowel} is the target vowel in the VNC sequence.
\item
  \texttt{NC} is the NC sequence.
\item
  \texttt{voicing} indicates the voicing of C.
\item
  \texttt{nas\_prop} is the proportion of coarticulatory nasalisation of
  the vowel.
\end{itemize}

\begin{figure}

\centering{

\pandocbounded{\includegraphics[keepaspectratio]{manuscript_files/figure-pdf/fig-nasal-1.pdf}}

}

\caption{\label{fig-nasal}Proportion of coarticulatory nasalisation
during the vowel in VNC sequences in German, depending on C voicing.}

\end{figure}%

Figure~\ref{fig-nasal} shows the proportion of coarticulatory
nasalisation in vowels followed by /nd/ (voiced) vs /nt/ (voiceless)
sequences, for the short vowels /i, e, a, o, u/. We can see a pattern of
higher nasalisation proportion in vowels followed by /nt/, at least in
the vowels /a, i, o/. For /e, u/, the distribution of nasalisation
proportion seems to be similar between the voiced and voiceless
contexts.

Now onto modelling with a beta regression. Note that a full appropriate
model would include further predictors (both constant and varying), but
for simplicity here we include only the following predictors: voicing
(voiced /nd/ vs voiceless /nt/), vowel (/i, e, a, o, u/), including an
interaction between them. As varying terms, we include a varying
intercept by speaker and a by-speaker varying slope for voicing and
vowel in interaction. As with the model from the first case study,
voicing and vowel are coded using indexing, by suppressing the intercept
with \texttt{0\ +}. Here's the code of the model.

\begin{Shaded}
\begin{Highlighting}[]
\NormalTok{nas\_prop\_bm }\OtherTok{\textless{}{-}} \FunctionTok{brm}\NormalTok{(}
\NormalTok{  nas\_prop }\SpecialCharTok{\textasciitilde{}} \DecValTok{0} \SpecialCharTok{+}\NormalTok{ voicing}\SpecialCharTok{:}\NormalTok{vowel }\SpecialCharTok{+}\NormalTok{ (}\DecValTok{0} \SpecialCharTok{+}\NormalTok{ voicing}\SpecialCharTok{:}\NormalTok{vowel }\SpecialCharTok{|}\NormalTok{ speaker),}
  \AttributeTok{data =}\NormalTok{ nasal,}
  \AttributeTok{family =}\NormalTok{ Beta,}
  \AttributeTok{cores =} \DecValTok{4}\NormalTok{,}
  \AttributeTok{seed =} \DecValTok{3749}\NormalTok{,}
  \AttributeTok{file =} \StringTok{"data/cache/nas\_prop\_bm"}
\NormalTok{)}
\end{Highlighting}
\end{Shaded}

Let's inspect the output of \texttt{fixef()}.

\begin{Shaded}
\begin{Highlighting}[]
\FunctionTok{round}\NormalTok{(}
  \FunctionTok{fixef}\NormalTok{(nas\_prop\_bm, }\AttributeTok{prob =} \FunctionTok{c}\NormalTok{(}\FloatTok{0.1}\NormalTok{, }\FloatTok{0.9}\NormalTok{)),}
  \AttributeTok{digits =} \DecValTok{2}
\NormalTok{)}
\end{Highlighting}
\end{Shaded}

\begin{verbatim}
                        Estimate Est.Error   Q10   Q90
voicingvoiced:vowela       -0.61      0.12 -0.77 -0.46
voicingvoiceless:vowela    -0.08      0.09 -0.19  0.04
voicingvoiced:vowele       -0.19      0.16 -0.38  0.01
voicingvoiceless:vowele    -0.25      0.10 -0.38 -0.12
voicingvoiced:voweli       -0.74      0.19 -0.98 -0.51
voicingvoiceless:voweli    -0.12      0.18 -0.35  0.10
voicingvoiced:vowelo       -0.34      0.16 -0.54 -0.14
voicingvoiceless:vowelo     0.25      0.15  0.05  0.44
voicingvoiced:vowelu       -0.71      0.18 -0.94 -0.49
voicingvoiceless:vowelu    -0.58      0.16 -0.78 -0.38
\end{verbatim}

Negative log-odds indicate a proportion that is smaller than 50\%, while
positive log-odds a proportion that is greater than 50\%. Generally, the
expected log-odd predictions are negative, indicating an overall
tendency for the nasalisation to take less than 50\% of the duration of
the vowel. Moreover, the predictions are higher for voiceless NC
sequences than for voiced NC sequences, indicating a greater proportion
of nasalisation in the former. However there is vowel-specific
variation, and there doesn't seem to be much of a difference in
nasalisation proportion in /e/ and /u/. Plotting the expected
predictions with \texttt{conditional\_effects()} should make this
clearer.

\begin{Shaded}
\begin{Highlighting}[]
\FunctionTok{conditional\_effects}\NormalTok{(nas\_prop\_bm, }\StringTok{"vowel:voicing"}\NormalTok{)}
\end{Highlighting}
\end{Shaded}

\pandocbounded{\includegraphics[keepaspectratio]{manuscript_files/figure-pdf/nas-prop-1.pdf}}

Now that we fitted the model we can use the draws to answer the two
research questions (repeated from above):

\begin{enumerate}
\def\labelenumi{\arabic{enumi}.}
\tightlist
\item
  Is the nasalisation proportion, on average across vowels, greater in
  voiceless NC sequences?
\item
  Is there individual speaker variation?
\end{enumerate}

To answer question 1, we can calculate the average difference in
nasalisation proportion by first calculating the average nasalisation
across all vowels for voiced and voiceless sequences and then take the
difference of those, similarly to what we have done in the Case Study 1
above.

\begin{Shaded}
\begin{Highlighting}[]
\CommentTok{\# extract only coefficient variables starting with "b\_"}
\NormalTok{nas\_prop\_bm\_draws }\OtherTok{\textless{}{-}} \FunctionTok{as\_draws\_df}\NormalTok{(nas\_prop\_bm, }\AttributeTok{variable =} \StringTok{"\^{}b\_"}\NormalTok{, }\AttributeTok{regex =} \ConstantTok{TRUE}\NormalTok{)}

\NormalTok{nas\_prop\_bm\_draws\_long }\OtherTok{\textless{}{-}}\NormalTok{ nas\_prop\_bm\_draws }\SpecialCharTok{|\textgreater{}} 
  \CommentTok{\# pivot vowel:c2 columns}
  \FunctionTok{pivot\_longer}\NormalTok{(}\StringTok{\textasciigrave{}}\AttributeTok{b\_voicingvoiced:vowela}\StringTok{\textasciigrave{}}\SpecialCharTok{:}\StringTok{\textasciigrave{}}\AttributeTok{b\_voicingvoiceless:vowelu}\StringTok{\textasciigrave{}}\NormalTok{, }\AttributeTok{names\_to =} \StringTok{"coeff"}\NormalTok{) }\SpecialCharTok{|\textgreater{}}
  \CommentTok{\# separate "coeff" labels into type ("b"), vowel and c2}
  \FunctionTok{separate}\NormalTok{(coeff, }\AttributeTok{into =} \FunctionTok{c}\NormalTok{(}\StringTok{"type"}\NormalTok{, }\StringTok{"voicing"}\NormalTok{, }\StringTok{"vowel"}\NormalTok{))}
\end{Highlighting}
\end{Shaded}

\begin{verbatim}
Warning: Dropping 'draws_df' class as required metadata was removed.
\end{verbatim}

\begin{Shaded}
\begin{Highlighting}[]
\NormalTok{nas\_prop\_bm\_draws\_long}
\end{Highlighting}
\end{Shaded}

\begin{verbatim}
# A tibble: 40,000 x 7
   .chain .iteration .draw type  voicing          vowel    value
    <int>      <int> <int> <chr> <chr>            <chr>    <dbl>
 1      1          1     1 b     voicingvoiced    vowela -0.463 
 2      1          1     1 b     voicingvoiceless vowela -0.0944
 3      1          1     1 b     voicingvoiced    vowele -0.155 
 4      1          1     1 b     voicingvoiceless vowele -0.208 
 5      1          1     1 b     voicingvoiced    voweli -0.481 
 6      1          1     1 b     voicingvoiceless voweli -0.0529
 7      1          1     1 b     voicingvoiced    vowelo -0.310 
 8      1          1     1 b     voicingvoiceless vowelo  0.321 
 9      1          1     1 b     voicingvoiced    vowelu -0.620 
10      1          1     1 b     voicingvoiceless vowelu -0.368 
# i 39,990 more rows
\end{verbatim}

Now let's calculate the mean nasalisation proportion within each draw by
voicing, and plot the resulting posterior distributions. Note that, as
discussed for Case study 1, when working wit log-odds it is important to
first do all necessary calculations in log-odds, here calculate the mean
log-odds across vowels, and \emph{then} transform the calculated
estimands to proportions/probabilities.

\begin{Shaded}
\begin{Highlighting}[]
\NormalTok{nas\_prop\_bm\_draws\_long\_voicing }\OtherTok{\textless{}{-}}\NormalTok{ nas\_prop\_bm\_draws\_long }\SpecialCharTok{|\textgreater{}} 
  \CommentTok{\# grouping by .draw and voicing ensures that averaging applies only within draw and voicing}
  \FunctionTok{group\_by}\NormalTok{(.draw, voicing) }\SpecialCharTok{|\textgreater{}} 
  \FunctionTok{summarise}\NormalTok{(}
    \CommentTok{\# calculate the mean value within draw/voicing in log{-}odds}
    \AttributeTok{value\_mean =} \FunctionTok{mean}\NormalTok{(value),}
    \CommentTok{\# we can now transform log{-}odds to proportion with plogis()}
    \AttributeTok{value\_mean\_prop =} \FunctionTok{plogis}\NormalTok{(value\_mean),}
    \AttributeTok{.groups =} \StringTok{"drop"}
\NormalTok{  )}

\NormalTok{nas\_prop\_bm\_draws\_long\_voicing}
\end{Highlighting}
\end{Shaded}

\begin{verbatim}
# A tibble: 8,000 x 4
   .draw voicing          value_mean value_mean_prop
   <int> <chr>                 <dbl>           <dbl>
 1     1 voicingvoiced       -0.406            0.400
 2     1 voicingvoiceless    -0.0804           0.480
 3     2 voicingvoiced       -0.391            0.404
 4     2 voicingvoiceless    -0.123            0.469
 5     3 voicingvoiced       -0.444            0.391
 6     3 voicingvoiceless    -0.132            0.467
 7     4 voicingvoiced       -0.439            0.392
 8     4 voicingvoiceless    -0.0982           0.475
 9     5 voicingvoiced       -0.383            0.405
10     5 voicingvoiceless    -0.266            0.434
# i 7,990 more rows
\end{verbatim}

\begin{Shaded}
\begin{Highlighting}[]
\NormalTok{nas\_prop\_bm\_draws\_long\_voicing }\SpecialCharTok{|\textgreater{}} 
  \FunctionTok{ggplot}\NormalTok{(}\FunctionTok{aes}\NormalTok{(value\_mean\_prop, voicing)) }\SpecialCharTok{+}
  \FunctionTok{stat\_halfeye}\NormalTok{(}\AttributeTok{.width =} \FunctionTok{c}\NormalTok{(}\FloatTok{0.6}\NormalTok{, }\FloatTok{0.8}\NormalTok{))}
\end{Highlighting}
\end{Shaded}

\pandocbounded{\includegraphics[keepaspectratio]{manuscript_files/figure-pdf/nas-prop-bm-draws-long-voicing-plot-1.pdf}}

The plot suggests an overall greater nasalisation proportion in
voiceless NC sequences. Let's quantify how greater as we did in Case
Study 1. We will use odd-ratios in this context as well, i.e.~we will
convert log-odds to odd-ratios using the \texttt{exp()} (exponential)
function (and as before we first calculate the difference and then
exponentiate the resulting values, after which we can take summary
measures, like means and quantile-based measures such as CrIs).

\begin{Shaded}
\begin{Highlighting}[]
\NormalTok{nas\_prop\_bm\_diff }\OtherTok{\textless{}{-}}\NormalTok{ nas\_prop\_bm\_draws\_long\_voicing }\SpecialCharTok{|\textgreater{}} 
  \CommentTok{\# pivot data to create one column per voicing with the mean drawn values,}
  \CommentTok{\# with one draw per raw. we need to drop the value\_mean\_prop col}
  \FunctionTok{select}\NormalTok{(}\SpecialCharTok{{-}}\NormalTok{value\_mean\_prop) }\SpecialCharTok{|\textgreater{}} 
  \FunctionTok{pivot\_wider}\NormalTok{(}\AttributeTok{names\_from =}\NormalTok{ voicing, }\AttributeTok{values\_from =}\NormalTok{ value\_mean) }\SpecialCharTok{|\textgreater{}} 
  \FunctionTok{mutate}\NormalTok{(}
    \CommentTok{\# calculate the difference of voiceless and voiced in log{-}odds}
    \AttributeTok{voicing\_diff =}\NormalTok{ voicingvoiceless }\SpecialCharTok{{-}}\NormalTok{ voicingvoiced,}
    \CommentTok{\# now transform with exp() to get the ratio difference}
    \AttributeTok{voicing\_diff\_ratio =} \FunctionTok{exp}\NormalTok{(voicing\_diff)}
\NormalTok{  )}
\NormalTok{nas\_prop\_bm\_diff}
\end{Highlighting}
\end{Shaded}

\begin{verbatim}
# A tibble: 4,000 x 5
   .draw voicingvoiced voicingvoiceless voicing_diff voicing_diff_ratio
   <int>         <dbl>            <dbl>        <dbl>              <dbl>
 1     1        -0.406          -0.0804        0.325               1.38
 2     2        -0.391          -0.123         0.268               1.31
 3     3        -0.444          -0.132         0.312               1.37
 4     4        -0.439          -0.0982        0.340               1.41
 5     5        -0.383          -0.266         0.117               1.12
 6     6        -0.541          -0.0903        0.450               1.57
 7     7        -0.512          -0.111         0.401               1.49
 8     8        -0.493          -0.152         0.341               1.41
 9     9        -0.527          -0.175         0.352               1.42
10    10        -0.491          -0.257         0.234               1.26
# i 3,990 more rows
\end{verbatim}

\begin{Shaded}
\begin{Highlighting}[]
\NormalTok{nas\_prop\_bm\_diff }\SpecialCharTok{|\textgreater{}} 
  \FunctionTok{reframe}\NormalTok{(}
    \CommentTok{\# 90\% CrI}
    \AttributeTok{q90 =} \FunctionTok{quantile2}\NormalTok{(voicing\_diff\_ratio, }\AttributeTok{probs =} \FunctionTok{c}\NormalTok{(}\FloatTok{0.05}\NormalTok{, }\FloatTok{0.95}\NormalTok{)),}
    \CommentTok{\# 80\% CrI}
    \AttributeTok{q80 =} \FunctionTok{quantile2}\NormalTok{(voicing\_diff\_ratio, }\AttributeTok{probs =} \FunctionTok{c}\NormalTok{(}\FloatTok{0.1}\NormalTok{, }\FloatTok{0.9}\NormalTok{)),}
    \CommentTok{\# 60\% CrI}
    \AttributeTok{q60 =} \FunctionTok{quantile2}\NormalTok{(voicing\_diff\_ratio, }\AttributeTok{probs =} \FunctionTok{c}\NormalTok{(}\FloatTok{0.2}\NormalTok{, }\FloatTok{0.8}\NormalTok{)),}
\NormalTok{  ) }\SpecialCharTok{|\textgreater{}} 
  \FunctionTok{mutate}\NormalTok{(}\FunctionTok{across}\NormalTok{(}\FunctionTok{everything}\NormalTok{(), }\SpecialCharTok{\textasciitilde{}}\FunctionTok{round}\NormalTok{(.x, }\DecValTok{2}\NormalTok{)))}
\end{Highlighting}
\end{Shaded}

\begin{verbatim}
# A tibble: 2 x 3
    q90   q80   q60
  <dbl> <dbl> <dbl>
1  1.23  1.27  1.33
2  1.69  1.63  1.56
\end{verbatim}

The CrIs of the ratio difference in nasalisation proportion in voiceless
vs voiced NC sequences suggest a robust increase of nasalisation in the
voiceless NC sequences, with a 90\% probability that the increase is
between 23\% and 69\% of the proportion in voiced NC sequences.

Moving onto question 2: is there individual speaker variation?
{[}TAMMINGA XXX{]} For this, we will use the \texttt{spread\_draws()}
function from tidybayes {[}XXX{]} to extract the draws of the varying
terms (in brms these are the coefficients that start with \texttt{r\_}).
There is quite a few steps of processing to get from the raw draws to
the estimand we need: while we have commented the following code, we
encourage readers to test each line sequentialy and inspect the
intermediate output to fully understand the process. We assume that
readers are familiar enough with models with varying terms (aka random
effects, mixed-effects models). What readers should note is that to
obtain the expected predictions of nasalisation proportion for each
speaker, the constant terms and the varying terms should be added (since
the varying terms indicate the deviation of each speaker from the
overall estimate).

\begin{Shaded}
\begin{Highlighting}[]
\FunctionTok{library}\NormalTok{(tidybayes)}

\NormalTok{nas\_prop\_r }\OtherTok{\textless{}{-}}\NormalTok{ nas\_prop\_bm }\SpecialCharTok{|\textgreater{}} 
  \CommentTok{\# extract draws from model, only \textasciigrave{}r\_speaker\textasciigrave{} varying terms}
  \FunctionTok{spread\_draws}\NormalTok{(r\_speaker[speaker,voicingvowel]) }\SpecialCharTok{|\textgreater{}} 
  \CommentTok{\# separate the column voicingvowel to two columns}
  \FunctionTok{separate}\NormalTok{(voicingvowel, }\FunctionTok{c}\NormalTok{(}\StringTok{"voicing"}\NormalTok{, }\StringTok{"vowel"}\NormalTok{)) }\SpecialCharTok{|\textgreater{}} 
  \CommentTok{\# join the draws with the \textasciigrave{}b\_\textasciigrave{} terms}
  \FunctionTok{left\_join}\NormalTok{(}\AttributeTok{y =}\NormalTok{ nas\_prop\_bm\_draws\_long) }\SpecialCharTok{|\textgreater{}} 
  \CommentTok{\# get the expected log{-}odd value of each speaker, in each draw}
  \CommentTok{\# this is the sum of the \textasciigrave{}value\textasciigrave{} from the b\_ terms and the value from the}
  \CommentTok{\# r\_speaker term.}
  \FunctionTok{mutate}\NormalTok{(}\AttributeTok{r\_speaker\_value =}\NormalTok{ value }\SpecialCharTok{+}\NormalTok{ r\_speaker) }\SpecialCharTok{|\textgreater{}} 
  \CommentTok{\# group the data for summarise}
  \FunctionTok{group\_by}\NormalTok{(.draw, speaker, voicing) }\SpecialCharTok{|\textgreater{}} 
  \CommentTok{\# get mean expected log{-}odds by draw, speaker and voicing (averaging across vowel)}
  \FunctionTok{summarise}\NormalTok{(}\AttributeTok{r\_speaker\_value\_mean =} \FunctionTok{mean}\NormalTok{(r\_speaker\_value)) }\SpecialCharTok{|\textgreater{}} 
  \CommentTok{\# make the data wider: two columns, one for voiced and one for voiceless}
  \FunctionTok{pivot\_wider}\NormalTok{(}\AttributeTok{names\_from =}\NormalTok{ voicing, }\AttributeTok{values\_from =}\NormalTok{ r\_speaker\_value\_mean) }\SpecialCharTok{|\textgreater{}} 
  \CommentTok{\# finally, calculate the difference in expected log{-}odds of voiceless and voiced}
  \FunctionTok{mutate}\NormalTok{(}\AttributeTok{voicing\_diff =}\NormalTok{ voicingvoiceless }\SpecialCharTok{{-}}\NormalTok{ voicingvoiced)}

\NormalTok{nas\_prop\_r}
\end{Highlighting}
\end{Shaded}

\begin{verbatim}
# A tibble: 140,000 x 5
# Groups:   .draw, speaker [140,000]
   .draw speaker voicingvoiced voicingvoiceless voicing_diff
   <int> <chr>           <dbl>            <dbl>        <dbl>
 1     1 S03          -1.42              0.0676      1.49   
 2     1 S04           0.209             0.185      -0.0238 
 3     1 S05          -0.284            -0.0546      0.229  
 4     1 S06           0.00669          -0.324      -0.331  
 5     1 S07          -0.446            -0.147       0.299  
 6     1 S08          -0.610            -0.127       0.483  
 7     1 S09          -0.473             0.0497      0.523  
 8     1 S10          -0.556            -0.311       0.245  
 9     1 S11           0.201            -0.426      -0.628  
10     1 S12          -0.0812           -0.0856     -0.00439
# i 139,990 more rows
\end{verbatim}

Figure~\ref{fig-nas-prop-r} plots the posterior distributions of the
expected log-odd difference of coarticulatory nasalisation in voiceless
vs voiced NC sequences (\emph{x}-axis), for each speaker in the data
(\emph{y}-axis), as predicted by the model. The red solid vertical line
indicates the constant (overall) expected log-odd difference based on
the (constant) \texttt{b\_} terms. The black dashed vertical line marks
log-difference 0 (i.e., no difference in proportion of nasalisation
between voiceless and voiced NC).

\begin{Shaded}
\begin{Highlighting}[]
\NormalTok{nas\_prop\_r }\SpecialCharTok{|\textgreater{}} 
  \FunctionTok{ggplot}\NormalTok{(}\FunctionTok{aes}\NormalTok{(voicing\_diff, }\FunctionTok{reorder}\NormalTok{(speaker, voicing\_diff))) }\SpecialCharTok{+}
  \FunctionTok{stat\_halfeye}\NormalTok{() }\SpecialCharTok{+}
  \FunctionTok{geom\_vline}\NormalTok{(}\AttributeTok{xintercept =} \FunctionTok{mean}\NormalTok{(nas\_prop\_bm\_diff}\SpecialCharTok{$}\NormalTok{voicing\_diff), }\AttributeTok{colour =} \StringTok{"red"}\NormalTok{) }\SpecialCharTok{+}
  \FunctionTok{geom\_vline}\NormalTok{(}\AttributeTok{xintercept =} \DecValTok{0}\NormalTok{, }\AttributeTok{linetype =} \StringTok{"dashed"}\NormalTok{) }\SpecialCharTok{+}
  \FunctionTok{lims}\NormalTok{(}\AttributeTok{x =} \FunctionTok{c}\NormalTok{(}\SpecialCharTok{{-}}\DecValTok{1}\NormalTok{, }\FloatTok{1.5}\NormalTok{))}
\end{Highlighting}
\end{Shaded}

\begin{figure}[H]

\centering{

\pandocbounded{\includegraphics[keepaspectratio]{manuscript_files/figure-pdf/fig-nas-prop-r-1.pdf}}

}

\caption{\label{fig-nas-prop-r}}

\end{figure}%

There is a lot of uncertainty within and between speakers: while the
distributions of most speakers are located in the positive range, some
expected distributions (see last 5 speakers at the bottom of figure) do
substantially span both negative and positive values. In other words,
while most speakers are more likely to have a larger nasalisation
proportion in voiceless NC sequences, a few might in fact have the
opposite pattern. Even among those speakers that do have a more robust
positive difference, there is a lot of uncertainty as to the magnitude
of the difference.


  \bibliography{linguistics.bib}



\end{document}
